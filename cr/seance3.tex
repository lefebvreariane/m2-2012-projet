\section{Troisième séance}

\subsection{Questions Réponses}

Le groupe a pu poser des questions quand aux données en entrée:
\begin{itemize}
    \item Possibilité d'avoir la pièce dans son état initial (avant pliage).\\

    \item Quel matériau sera utilisé pour les tests.\\
        Un acier, possédant des caractéristiques  classiques.
        C'est en fonction du modèle que l'on peut déterminer les caractéristiques nécessaires ou possibles.
    \item Faut il prendre en compte un jeu dans la position du poinçon à l'état final?\\
        L'épaisseur de la tôle étant supposée constante, le poinçon sera fixe (sans jeu) à la fin de son déplacement.
        Il faut cependant faire attention à une tolérance de l'ordre du centième de millimètre et aussi contrôler la cohérence du modèle (poinçon DANS la tôle à un instant t?)
\end{itemize}

Nous avons ensuite vu des notions mathématiques utiles dans le cadre du projet telles que les maillages à utiliser, les méthodes d'approximations pour la positions des points en 2D.

\subsection{À faire}
Il nous faut prendre connaissance du logiciel freemem++ qui permet de simuler les déformations.
Nous devons apprendre à nous en servir et trouver une façon propre de l'intégrer à notre projet.
%\begin{itemize}
    %\item Peut on avoir la pièce dans son état initial?\\
        %???
    %\item Un matériau de test?\\
        %Un acier.\\
        %Caractéristiques mécaniques classiques.
        %C'est en fonction du modèle qu'on peut savoir quelles caractéristiques on doit/peut choisir.
    %\item Faut il considérer un jeu dans la position du poinçon à la fin du mouvement?\\
        %Non, on considère l'épaisseur de la tôle constante donc le poinçon se déplace au maximum sans déformer la matériau. (Dimensions nominales)\\
        %En général, la tolérance est de l'ordre du centième de millimètre.\\

        %Il faut prendre en compte les différentes tolérances.\\
        %Savoir vérifier si le modèle est mauvais (Poinçons DANS la tôle après pliage).
%\end{itemize}

%Mécanique des milieux déformables:
%\begin{itemize}
    %\item Utiliser un maillage triangulaire.
        %Modèle à raffiner selon la précision requise pour l'affichage. (Poisson, Laplace, différences finies)
%\end{itemize}

%Interpolation de Lagrange.
%Intégrale de Riemann.
%Équation de Poisson.
%$\Delta u$ : Laplacien de u (Somme des dérivées). $\Delta u = \frac{\delta^2 u}{x^2} + \frac{\delta^2 u}{y^2}$
%On a besoin de: $f(u) = f(\nable u, u, \Delta u)$
%Ce qu'il reste après une intégration par partie: $\intD(u) \nabla u \nable v d$

%Méthode de Newton: calculer le $x$ tel que $f(x)=0$ par une suite de tangentes.

%Freefem++

%Fibre neutre
%Beam

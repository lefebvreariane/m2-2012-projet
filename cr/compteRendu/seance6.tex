 %open fem
 %p48 -> 50 Déformation en fonction du déplacement 
\section{Séance 6}

\subsection{Formules}
\subsubsection{Position du poinçon au cours du temps}
$P(t) = \frac{-Dmax}{2}*\cos(\frac{2}{Tmax}*\pi*t)+\frac{Dmax}{2}$

\subsubsection{Vitesse au cours du temps}
$V(t) = \frac{\pi*Dmax}{Tmax}*\sin(\frac{2*\pi}{Tmax}*t)$

\subsubsection{Accélération au cours du temps}
$A(t) = \frac{2*\pi^2*Dmax}{Tmax^2}*\cos(\frac{2*\pi}{Tmax}*t)$

\subsection{Forces en jeu}
À partir des mouvements données par les formules ci dessus nous pouvons calculer les foces en jeu.
Les équations sont données dans une thèse (Mohamed AZAOUZI - 2007) et décrivent les quatres calculs de forces (p55):
\begin{itemize}
    \item Loi d'écrouissage d'Hollomon.
    \item Loi d'écrouissage de Krupkowski.
    \item Loi d'écrouissage de Ludwick.
    \item Loi d'écrouissage de Voce.
\end{itemize}
Nous avons aussi les caractéristiques d'un matérieux pour les tests unitaires (p56).

\subsection{Changements}
Le logiciel initialement prévu pour les calculs de déformations (freefem++) ne gère pas les éléments finis "coque".
Nous utiliserons donc la librairie \textbf{Open fem} dans scilab pour le remplacer.

\section{Séance 4}

\subsection{Interface utilisateur}
Nous avons choisi que l'utilisateur détermine le temps de visualisation de la déformation ce qui nous donnera le nombre d'étapes à calculer pour une scène.
Une étape correspond à la déformation à un temps t.

Le programme utilise Qt pour son interface graphique.
Il sera possible d'ouvrir plusieurs scènes en même temps (un thread par scène).

\subsection{Architecture}
Notre programme utilisera le logiciel freefem++ pour les calculs de déformations de la tôle au cours du temps.
Le calcul de collisions entre la tôle, la matrice et le poinçon sera effectué pour chaque étape de la cinématique.
Aucune déformation n'est effectuée tant qu'il n'y a pas de collision détectée.
Une étape correspondra donc uniquement à la descente du poinçon jusqu'au premier contact avec la tôle.

Nous supposons une épaisseur constante pour la tôle et travaillons à partir de la fibre neutre pour l'affichage et les calculs de collisions.

\subsection{Calcul d'une déformation}

\paragraph{Entrées}
\begin{itemize}
    \item Les forces appliquées par le poinçon sur la tôle.
    \item Les points de la fibre neutre à l'instant t-1 ainsi que son épaisseur.
    \item Les contraintes sur certains points (fixes ou non).
    \item Le polygone de la matrice.
\end{itemize}

\paragraph{Sorties}
\begin{itemize}
    \item La nouvelle position des points de la fibre neutre à l'instant t.
\end{itemize}

\subsection{À définir}
\begin{itemize}
    \item Affichage de l'aire totale de déplacement de la tôle ou sélection d'un (ou plusieurs) point(s) par l'utilisateur et affichage de son déplacement au cours du temps.
    \item Types de données échangées avec freefem++: nurbs, points, structures complexes.
    \item Enregistrement ou non de chaque étape sur disque dur pour ré affichage rapide.
    \item Possibilité de calculer une suite de scène si le temps imparti est suffisant.
\end{itemize}

%Le but du projet est de donner les informations utiles à la réduction de l'encombrement de la machine outils.

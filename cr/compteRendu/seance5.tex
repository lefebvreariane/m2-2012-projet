\section{Séance 5}

\subsection{Interface utilisateur}
L'utilisateur ne choisira pas seulement une durée de visualisation mais:
\begin{itemize}
    \item Un pas de déplacement pour le poinçon.
    \item Une durée de visualisation.
    \item Une durée de pliage.
\end{itemize}
Le pas de déplacement correspondra à la précision de notre simulation et il sera possible de revoir la cinématique intégralement ou image par image.

Il sera possible de voir la surface parcourue par la tôle durant toute une scène et de suivre un ou plusieurs points au cours du temps.

\subsection{Représentation des scènes}
Chaque scène sera décrite par un fichier xml.
Elle contiendra la position initiale de la tôle et de la matrice ainsi que l'ensemble des positions du poinçon.
Il sera donc possible de simuler des mouvements simples de translation ou plus complexes tels que des rotations pour le poinçon.

\subsection{Utilisation de freefem++}
La durée de pliage et le déplacement du poinçon permettront d'évaluer une vitesse de déplacement (position, vitesse, accélération).
Ces informations seront passées à freefem++ pour le calcul de la déformation de la tôle.

\subsection{À définir}
\begin{itemize}
    \item Comment évaluer le point à suivre?\\
        Clic souris ou noeud dans le maillage de la tôle.
    \item Comment avoir des forces réalistes à passer en argument à freefem++?
    \item Quelle quantité d'informations sauvegarder sur disque ou garde en RAM pour chaque scène?
\end{itemize}
